%!TEX TS-program = xelatex
%!TEX encoding = UTF-8 Unicode

\documentclass[10pt, a4paper]{article}
\usepackage{fontspec} 

% DOCUMENT LAYOUT
\usepackage{geometry} 
\geometry{a4paper, textwidth=5.5in, textheight=8.5in, marginparsep=7pt, marginparwidth=.6in}
\setlength\parindent{0in}

% FONTS
\usepackage[usenames,dvipsnames]{xcolor}
\usepackage{xunicode}
\usepackage{xltxtra}
\defaultfontfeatures{Mapping=tex-text}
%\setromanfont [Ligatures={Common}, Numbers={OldStyle}, Variant=01]{Linux Libertine O}
%\setmonofont[Scale=0.8]{Monaco}
%%% modified by Karol Kozioł for ShareLaTeX use
\setmainfont[
  Ligatures={Common}, Numbers={OldStyle}, Variant=01,
  BoldFont=LinLibertine_RB.otf,
  ItalicFont=LinLibertine_RI.otf,
  BoldItalicFont=LinLibertine_RBI.otf
]{LinLibertine_R.otf}
\setmonofont[Scale=0.8]{DejaVuSansMono.ttf}

% ---- CUSTOM COMMANDS
\chardef\&="E050
\newcommand{\html}[1]{\href{#1}{\scriptsize\textsc{[html]}}}
\newcommand{\pdf}[1]{\href{#1}{\scriptsize\textsc{[pdf]}}}
\newcommand{\doi}[1]{\href{#1}{\scriptsize\textsc{[doi]}}}
% ---- MARGIN YEARS
\usepackage{marginnote}
\newcommand{\amper{}}{\chardef\amper="E0BD }
\newcommand{\years}[1]{\marginnote{\scriptsize #1}}
\renewcommand*{\raggedleftmarginnote}{}
\setlength{\marginparsep}{7pt}
\reversemarginpar

% HEADINGS
\usepackage{sectsty} 
\usepackage[normalem]{ulem} 
\sectionfont{\mdseries\upshape\Large}
\subsectionfont{\mdseries\scshape\normalsize} 
\subsubsectionfont{\mdseries\upshape\large} 

% PDF SETUP
% ---- FILL IN HERE THE DOC TITLE AND AUTHOR
\usepackage[%dvipdfm, 
bookmarks, colorlinks, breaklinks, 
% ---- FILL IN HERE THE TITLE AND AUTHOR
	pdftitle={Andrea Fioraldi CV},
	pdfauthor={Andrea Fioraldi},
	pdfproducer={http://nitens.org/taraborelli/cvtex}
]{hyperref}  
\hypersetup{linkcolor=blue,citecolor=blue,filecolor=black,urlcolor=MidnightBlue} 

\renewcommand{\baselinestretch}{1.15} 

% DOCUMENT
\begin{document}
{\LARGE Andrea Fioraldi}\\[1cm]
{\em Quickly translated and incomplete copy of the curriculum vitae in Italian, please refer to the English curriculum for a complete version.} \\[0.4cm]
 Via [REDACTED], [REDACTED]\\
Pontinia, LT. \texttt{04014}
Italia.\\[.2cm]
Telefono: \texttt{+39 [REDACTED]}\\[.2cm]
email: \href{mailto:andreafioraldi@gmail.com}{andreafioraldi@gmail.com}\\
blog: \href{https://andreafioraldi.github.io}{andreafioraldi.github.io}\\
\textsc{github}: \href{https://github.com/andreafioraldi}{andreafioraldi}\\ 
\textsc{twitter}: \href{https://twitter.com/andreafioraldi}{@andreafioraldi}\\

\par\vspace{1cm}

Nato:  [REDACTED], 1996---[REDACTED], Italy\\
Nazionalità:  Italiana

%%\hrule
\section*{Posizione corrente}
\emph{Studente magistrale in Engineering in Computer Science}, Sapienza Università di Roma.

%%\hrule
\section*{Area di specializzazione}
 System Security • Binary analysis and exploitation

%\hrule
\section*{Riconoscimenti e premi}
\noindent
\years{2017}Primo posto in Binary Analysis (ex aequo) a CyberChallenge.IT 17.

\years{2017}Secondo posto in Penetration Testing a CyberChallenge.IT 17.

\years{2017}Terzo posto con il Team Nazionale di Cybersecurity alla European CyberSecurity Challenge (ECSC).

\years{2018}Settimo post al DEF CON 26 CTF (finale mondiale) con il team \href{https://mhackeroni.it}{mHACKeroni}.

\years{2018}Primo posto al Chaos Communication Congress CTF con il team \href{https://mhackeroni.it}{mHACKeroni}+KJC.

\years{2019}Quinti posto al DEF CON 27 CTF (finale mondiale) con il team \href{https://mhackeroni.it}{mHACKeroni}.

\section*{Laurea triennale}

Ho ottenuto una laurea triennale in Ingegneria Informatica e Automatica all'università Sapienza di Roma con voto 110/110 e lode.

La mia tesi è "Symbolic Execution and Debugging Synchronization", disponibile su \href{https://www.researchgate.net/publication/327655380_Symbolic_Execution_and_Debugging_Synchronization}{ResearchGate}

\section*{Progetti rilevanti}
\subsection*{Organizzazioni ed eventi}
\years{2017}Co-fondatore del team accademico di Capture The Flag \href{https://theromanxpl0it.github.io}{TheRomanXpl0it} (TRX).

\years{2018}Co-fondatore del mega-team italiano di Capture The Flag \href{https://mhackeroni.it}{mHACKeroni}.

\years{2018}Fondatore del \href{https://defcon11396.it}{DEF CON 11396 Rome}, un gruppo che si riunisce ogni mese per talk su argomenti avanzati di computer security al di fuori del comune piano di studi.

\subsection*{Software}
\years{2017-2018}\href{https://carbonara-project.github.io/}{\emph{Carbonara}}, a malware research platform designed to recognize duplicated functions between binaries at scale and speed-up the static malware analysis process

\years{2018}\href{https://github.com/andreafioraldi/angrdbg}{\emph{angrdbg}}, an abstract library used to implement synchronization between a concrete execution environment (tipically a debugger) and the \href{http://angr.io/}{angr} symbolic execution engine

\years{2018}\href{https://andreafioraldi.github.io/IDAngr/}{\emph{IDAngr}}, an IDA Pro debugger plugin that implements the angrdbg API in IDA with an user-friendly GUI

\years{2018}\href{https://github.com/andreafioraldi/angrgbg}{\emph{angrgdb}}, create an angr state from the current GDB state on top of angrdbg

\years{2019}\href{https://github.com/vanhauser-thc/AFLplusplus}{\emph{AFL++}}, AFL 2.56b with community patches, AFLfast power schedules, QEMU 3.1 upgrade + laf-intel support, MOpt mutators, InsTrim instrumentation, Unicorn mode and a lot more! 

\section*{Interessi di ricerca}

Il mio principale interesse in security è scrivere strumenti per automatizzare la Binary Analysis
usando tecniche avanzate come l'Esecuzione Simbolica, l'Istrumentazione Dinamica dei binari,
il Taint Tracking e altro.

Ad'ora ricerco nuove metodologie per lo Smart Fuzzing.

Sono bravo nella ricerca di vulnerabilità, nella binary exploitation e nella malware analysis
poichè sono i topic di applicazione della mia attività di ricerca.

Sono anche un entusiasta della teoria dei linguaggi di programmazione.
Ho sviluppato diversi linguaggi di programmazione educazionali sia per la mia
stessa educazione che per l'educazione del giovani compagni di team in TRX
riguardo le vulnerabilità e l'exploitation dei Linguaggi di Programmazione.

\section*{Competenze tecniche}

\subsection*{Linguaggi di programmazione}

Sono molto proficuo con il C, C++, Python, ASM x86 ed ho familiarità con Javascript, Java, Scala, C\#.

\subsection*{Sistemi operativi}

Sono bravo nello scrivere codice per sistemi POSIX e non male con MS Windows (sviluppo userspace).
Conosco gli internals del sistema GNU/Linux e ho familiaritò con la codebase del kernel.

\subsection*{Strumenti di secruity}

Conosco molto bene come usare IDA Pro, GDB, Frida, ed Intel PIN ad, inoltre, so usare e conosco il funzionamento interno di American Fuzzy Lop, QEMU TCG ed angr.
Ho esperienza limitata con radare2.

\section*{Volontariato}
\noindent
\years{2018}Ho insegnato Binary Exploitation a CyberChallenge.IT 2018 e 2019 a Roma.

\section*{Hobbies and Passions}
Tromba • Birra fatta in casa • Trekking • Mountain bike

%\vspace{1cm}
\vfill{}
%\hrulefill

\begin{center}
{\scriptsize  Last updated: \today\- •\- 
% ---- PLEASE LEAVE THIS BACKLINK FOR ATTRIBUTION AS PER CC-LICENSE
Typeset in \href{http://nitens.org/taraborelli/cvtex}{
%\fontspec{Times New Roman}
\XeTeX }\\
% ---- FILL IN THE FULL URL TO YOUR CV HERE
\href{https://raw.githubusercontent.com/andreafioraldi/CurriculumVitae/master/cv.pdf}{https://raw.githubusercontent.com/andreafioraldi/CurriculumVitae/master/cv.pdf}}
\end{center}

{\scriptsize Italian legal note: Autorizzo il trattamento dei miei dati personali presenti nel cv ai sensi del Decreto Legislativo 30 giugno 2003, n. 196 “Codice in materia di protezione dei dati personali” e del GDPR (Regolamento UE 2016/679)}

\end{document}
